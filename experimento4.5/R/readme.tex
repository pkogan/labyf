% Options for packages loaded elsewhere
\PassOptionsToPackage{unicode}{hyperref}
\PassOptionsToPackage{hyphens}{url}
%
\documentclass[
]{article}
\usepackage{amsmath,amssymb}
\usepackage{iftex}
\ifPDFTeX
  \usepackage[T1]{fontenc}
  \usepackage[utf8]{inputenc}
  \usepackage{textcomp} % provide euro and other symbols
\else % if luatex or xetex
  \usepackage{unicode-math} % this also loads fontspec
  \defaultfontfeatures{Scale=MatchLowercase}
  \defaultfontfeatures[\rmfamily]{Ligatures=TeX,Scale=1}
\fi
\usepackage{lmodern}
\ifPDFTeX\else
  % xetex/luatex font selection
\fi
% Use upquote if available, for straight quotes in verbatim environments
\IfFileExists{upquote.sty}{\usepackage{upquote}}{}
\IfFileExists{microtype.sty}{% use microtype if available
  \usepackage[]{microtype}
  \UseMicrotypeSet[protrusion]{basicmath} % disable protrusion for tt fonts
}{}
\makeatletter
\@ifundefined{KOMAClassName}{% if non-KOMA class
  \IfFileExists{parskip.sty}{%
    \usepackage{parskip}
  }{% else
    \setlength{\parindent}{0pt}
    \setlength{\parskip}{6pt plus 2pt minus 1pt}}
}{% if KOMA class
  \KOMAoptions{parskip=half}}
\makeatother
\usepackage{xcolor}
\usepackage[margin=1in]{geometry}
\usepackage{color}
\usepackage{fancyvrb}
\newcommand{\VerbBar}{|}
\newcommand{\VERB}{\Verb[commandchars=\\\{\}]}
\DefineVerbatimEnvironment{Highlighting}{Verbatim}{commandchars=\\\{\}}
% Add ',fontsize=\small' for more characters per line
\usepackage{framed}
\definecolor{shadecolor}{RGB}{248,248,248}
\newenvironment{Shaded}{\begin{snugshade}}{\end{snugshade}}
\newcommand{\AlertTok}[1]{\textcolor[rgb]{0.94,0.16,0.16}{#1}}
\newcommand{\AnnotationTok}[1]{\textcolor[rgb]{0.56,0.35,0.01}{\textbf{\textit{#1}}}}
\newcommand{\AttributeTok}[1]{\textcolor[rgb]{0.13,0.29,0.53}{#1}}
\newcommand{\BaseNTok}[1]{\textcolor[rgb]{0.00,0.00,0.81}{#1}}
\newcommand{\BuiltInTok}[1]{#1}
\newcommand{\CharTok}[1]{\textcolor[rgb]{0.31,0.60,0.02}{#1}}
\newcommand{\CommentTok}[1]{\textcolor[rgb]{0.56,0.35,0.01}{\textit{#1}}}
\newcommand{\CommentVarTok}[1]{\textcolor[rgb]{0.56,0.35,0.01}{\textbf{\textit{#1}}}}
\newcommand{\ConstantTok}[1]{\textcolor[rgb]{0.56,0.35,0.01}{#1}}
\newcommand{\ControlFlowTok}[1]{\textcolor[rgb]{0.13,0.29,0.53}{\textbf{#1}}}
\newcommand{\DataTypeTok}[1]{\textcolor[rgb]{0.13,0.29,0.53}{#1}}
\newcommand{\DecValTok}[1]{\textcolor[rgb]{0.00,0.00,0.81}{#1}}
\newcommand{\DocumentationTok}[1]{\textcolor[rgb]{0.56,0.35,0.01}{\textbf{\textit{#1}}}}
\newcommand{\ErrorTok}[1]{\textcolor[rgb]{0.64,0.00,0.00}{\textbf{#1}}}
\newcommand{\ExtensionTok}[1]{#1}
\newcommand{\FloatTok}[1]{\textcolor[rgb]{0.00,0.00,0.81}{#1}}
\newcommand{\FunctionTok}[1]{\textcolor[rgb]{0.13,0.29,0.53}{\textbf{#1}}}
\newcommand{\ImportTok}[1]{#1}
\newcommand{\InformationTok}[1]{\textcolor[rgb]{0.56,0.35,0.01}{\textbf{\textit{#1}}}}
\newcommand{\KeywordTok}[1]{\textcolor[rgb]{0.13,0.29,0.53}{\textbf{#1}}}
\newcommand{\NormalTok}[1]{#1}
\newcommand{\OperatorTok}[1]{\textcolor[rgb]{0.81,0.36,0.00}{\textbf{#1}}}
\newcommand{\OtherTok}[1]{\textcolor[rgb]{0.56,0.35,0.01}{#1}}
\newcommand{\PreprocessorTok}[1]{\textcolor[rgb]{0.56,0.35,0.01}{\textit{#1}}}
\newcommand{\RegionMarkerTok}[1]{#1}
\newcommand{\SpecialCharTok}[1]{\textcolor[rgb]{0.81,0.36,0.00}{\textbf{#1}}}
\newcommand{\SpecialStringTok}[1]{\textcolor[rgb]{0.31,0.60,0.02}{#1}}
\newcommand{\StringTok}[1]{\textcolor[rgb]{0.31,0.60,0.02}{#1}}
\newcommand{\VariableTok}[1]{\textcolor[rgb]{0.00,0.00,0.00}{#1}}
\newcommand{\VerbatimStringTok}[1]{\textcolor[rgb]{0.31,0.60,0.02}{#1}}
\newcommand{\WarningTok}[1]{\textcolor[rgb]{0.56,0.35,0.01}{\textbf{\textit{#1}}}}
\usepackage{graphicx}
\makeatletter
\def\maxwidth{\ifdim\Gin@nat@width>\linewidth\linewidth\else\Gin@nat@width\fi}
\def\maxheight{\ifdim\Gin@nat@height>\textheight\textheight\else\Gin@nat@height\fi}
\makeatother
% Scale images if necessary, so that they will not overflow the page
% margins by default, and it is still possible to overwrite the defaults
% using explicit options in \includegraphics[width, height, ...]{}
\setkeys{Gin}{width=\maxwidth,height=\maxheight,keepaspectratio}
% Set default figure placement to htbp
\makeatletter
\def\fps@figure{htbp}
\makeatother
\setlength{\emergencystretch}{3em} % prevent overfull lines
\providecommand{\tightlist}{%
  \setlength{\itemsep}{0pt}\setlength{\parskip}{0pt}}
\setcounter{secnumdepth}{-\maxdimen} % remove section numbering
\ifLuaTeX
  \usepackage{selnolig}  % disable illegal ligatures
\fi
\IfFileExists{bookmark.sty}{\usepackage{bookmark}}{\usepackage{hyperref}}
\IfFileExists{xurl.sty}{\usepackage{xurl}}{} % add URL line breaks if available
\urlstyle{same}
\hypersetup{
  pdftitle={Theming with bslib and thematic},
  hidelinks,
  pdfcreator={LaTeX via pandoc}}

\title{Theming with bslib and thematic}
\author{}
\date{\vspace{-2.5em}}

\begin{document}
\maketitle

\hypertarget{r-markdown}{%
\subsection{R Markdown}\label{r-markdown}}

This is an R Markdown document themed with
\href{https://rstudio.github.io/bslib/}{\texttt{\{bslib\}} package}.
\texttt{\{bslib\}} makes it easy to customize the main colors and fonts
of a \texttt{html\_document},
\href{https://flexdashboard-pkg.netlify.app/articles/articles/theme.html}{\texttt{flexdashboard::flex\_dashboard}},
\href{https://shiny.rstudio.com/reference/shiny/latest/fluidPage.html}{shiny::fluidPage()},
or more generally any website that uses
\href{https://getbootstrap.com/}{Bootstrap} for styling. The
\texttt{theme} parameter in the yaml front-matter of this Rmd document
describes a
\href{https://rstudio.github.io/bslib/reference/bs_theme.html}{\texttt{bslib::bs\_theme()}}
object, which provides access to 100s of
\href{https://rstudio.github.io/bslib/articles/bs5-variables.html}{theming
options} (via its \texttt{...} argument) in addition to the main options
demonstrated here (e.g., \texttt{bg}, \texttt{fg}, \texttt{primary},
etc).

This particular example uses \texttt{\{bslib\}}'s default
\href{https://rstudio.github.io/bslib/articles/bslib.html\#versions}{Bootstrap
version (which, at the time of writing, is Bootstrap 5)}. However, if
reproducibility is important, it's recommended that you ``lock-in'' the
version by adding \texttt{version:\ 5} to the \texttt{theme} definition.

\hypertarget{themed-plots}{%
\subsection{Themed Plots}\label{themed-plots}}

When running this document with
\href{https://rstudio.github.io/thematic/}{\texttt{\{thematic\}}}
installed, the \texttt{thematic::thematic\_rmd(font\ =\ "auto")}
effectively translates \texttt{theme} (CSS) settings to new global
theming defaults for \texttt{\{ggplot2\}}, \texttt{\{lattice\}}, and
\texttt{\{base\}} R graphics:

\hypertarget{ggplot2}{%
\subsubsection{ggplot2}\label{ggplot2}}

\begin{Shaded}
\begin{Highlighting}[]
\FunctionTok{library}\NormalTok{(ggplot2)}

\FunctionTok{ggplot}\NormalTok{(mpg, }\FunctionTok{aes}\NormalTok{(displ, hwy)) }\SpecialCharTok{+}
  \FunctionTok{geom\_point}\NormalTok{() }\SpecialCharTok{+} \FunctionTok{geom\_smooth}\NormalTok{()}
\end{Highlighting}
\end{Shaded}

\begin{verbatim}
## Error : The fig.showtext code chunk option must be TRUE
\end{verbatim}

\begin{verbatim}
## `geom_smooth()` using method = 'loess' and formula = 'y ~ x'
\end{verbatim}

\includegraphics{readme_files/figure-latex/unnamed-chunk-1-1.pdf}

\hypertarget{lattice}{%
\subsubsection{lattice}\label{lattice}}

\begin{Shaded}
\begin{Highlighting}[]
\NormalTok{lattice}\SpecialCharTok{::}\FunctionTok{show.settings}\NormalTok{()}
\end{Highlighting}
\end{Shaded}

\begin{verbatim}
## Error : The fig.showtext code chunk option must be TRUE
\end{verbatim}

\includegraphics{readme_files/figure-latex/unnamed-chunk-2-1.pdf}

\hypertarget{base}{%
\subsubsection{base}\label{base}}

\begin{Shaded}
\begin{Highlighting}[]
\FunctionTok{plot}\NormalTok{(pressure, }\AttributeTok{col =}\NormalTok{ thematic}\SpecialCharTok{::}\FunctionTok{thematic\_get\_option}\NormalTok{(}\StringTok{"accent"}\NormalTok{))}
\end{Highlighting}
\end{Shaded}

\begin{verbatim}
## Error : The fig.showtext code chunk option must be TRUE
\end{verbatim}

\includegraphics{readme_files/figure-latex/unnamed-chunk-3-1.pdf}

\end{document}
